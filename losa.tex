\documentclass[notheorems]{beamer}
\usetheme{Madrid}
\usecolortheme{spruce}

\usepackage{graphicx} % Required for inserting images
\usepackage[romanian]{babel}
\usepackage[center]{caption}
\usepackage{float}
\usepackage{amsmath, amssymb}
\usepackage{mathtools}
\usepackage[T1]{fontenc}
\usepackage{ragged2e}
\usepackage[edges]{forest}

\theoremstyle{definition}
\newtheorem*{obs}{Observație.}
\newtheorem{theorem}{Teoremă.}
\newtheorem{proposition}{Propoziție.}
\newtheorem{corollary}{Corolar.}
\newtheorem{definition}{Definiție.}
\newtheorem{lemma}{Lemă.}

\AtBeginSection[]{
  \begin{frame}
  \vfill
  \centering
  \begin{beamercolorbox}[sep=8pt,center,shadow=true,rounded=true]{title}
    \usebeamerfont{title}\insertsectionhead\par%
  \end{beamercolorbox}
  \vfill
  \end{frame}
}

\newcommand{\E}{\mathbb{E}}

% \setbeamertemplate{items}[triangle]

\title[Șiruri audioactive]{\textbf{Șiruri audioactive}}
\author[Patrichi, Mureanu]{
    Răzvan-Anton Mureanu \texttt{\small<razvan.mureanu@cnmbct.ro>}
    \texorpdfstring{\\}{}
    Ștefan Patrichi \texttt{\small<stefan.patrichi.07@cnmbct.ro>}
}

\institute[CNMB CT]{
    \normalsize
    Colegiul Național „Mircea cel Bătrân” Constanța
}

\date[CITYINNOEDU]{AI + $\alpha$ + Z = Matematica noilor generații\\8 noiembrie 2025}

% Usual LaTeX font (math)
\usefonttheme[onlymath]{serif}

\begin{document}

% Usual LaTeX font (text)
\rmfamily

\justifying
\frame{\titlepage}

\Large
\begin{frame}
\frametitle{Care este următorul termen?}
\only<1>{
    \begin{center}
    1\\
    11\\
    21\\
    1211\\
    111221\\
    312211\\
    13112221\\
    1113213211\\
    \end{center}
}
\only<2>{
    \huge
    \begin{center}
        111221\\
        312211\\
    \end{center}
}
\only<3>{
    \huge
    \begin{center}
        \textcolor{blue}{111}\textcolor{red}{22}\textcolor{teal}{1}\\
        3\textcolor{blue}{1} 2\textcolor{red}{2} 1\textcolor{teal}{1}\\
    \end{center}
}
\end{frame}

\begin{frame}
\frametitle{Monoid liber}
    \begin{itemize}
        % nu lucram cu numere, ci cu cuvinte
        % notiunea de alfabet, cuvant, concatenare
        % notatie multiplicativa pt operatia de concatenare
        \item<1-> Alfabet: $\Sigma = \{1, 2, 3, 4, 5, 6, 7, 8, 9\}$
        \item<2-> Cuvânt: $\alpha = a_1a_2\ldots a_k$ cu $a_i \in \Sigma$
        \item<3-> Mulțimea cuvintelor: $\Sigma^* = \{a_1a_2\ldots a_k \mid a_i \in \Sigma, k \in \mathbb{N}^*\}$
        \item<4-> Lungimea cuvântului: $|\alpha| = k$
        \item<5-> Concatenarea: $\alpha = a_1a_2\ldots a_k$, $\beta = b_1b_2\ldots b_l$\\
        Atunci $\alpha\cdot\beta = \alpha\beta = a_1a_2\ldots a_kb_1b_2\ldots b_l$
        \item<6-> $(\Sigma^*, \cdot)$ = \textbf{monoidul liber} generat de mulțimea $\Sigma$
    \end{itemize}
\end{frame}

\begin{frame}
\frametitle{Convenții pentru notația multiplicativă}
\begin{itemize}
    \item<1-> Notație: $\displaystyle \underbrace{aa\ldots a}_{\text{de $m$ ori}}\underbrace{bb\ldots b}_{\text{de $n$ ori}} = a^m b^n$
    \item<2-> $a^ma^n = a^{m+n}$
    \item<3-> Exemplu: $111221 = 1^3 2^2 1^1$ \only<3-> {$\rightarrow 312211$}

\end{itemize}
\end{frame}

\begin{frame}
    \frametitle{Funcția de tranziție}
    \begin{itemize}
        \item<1-> $f : \Sigma^* \rightarrow \Sigma^*$, $\boxed{f(a_1^{n_1}a_2^{n_2}\ldots a_k^{n_k}) = n_1 a_1 n_2 a_2 \ldots n_k a_k}$
        \item<2-> Condiție: \textcolor{red}{$a_i \ne a_{i+1}$} $\left(a_i^m a_i^n = a_i^{m+n}\right)$

    \end{itemize}
\end{frame}

\begin{frame}
\frametitle{Proprietăți de bază}
\only<1>{
\begin{proposition}
    $\operatorname{Im} f = \{\beta \left|\right. \beta\in \Sigma^*, |\beta| = 2k, k\in\mathbb{N}^*\} 
    \stackrel{\text{not}}{=} (\Sigma^2)^*$
\end{proposition}
\begin{proof}
Pentru incluziunea $\subset$: Dacă $\alpha = a_1^{n_1}a_2^{n_2}\ldots a_k^{n_k}$, atunci $f(\alpha) = n_1a_1n_2a_2\ldots n_ka_k$, deci $|f(\alpha)| = 2k$. 

Pentru incluziunea $\supset$: Cuvântul $\beta = b_1b_2\ldots b_{2k}$ are inversul (unic)
$\alpha$ = $b_2^{b_1}\ldots b_{2k}^{b_{2k-1}}$.
\end{proof}
}
\only<2>{
\begin{obs}
$g : \Sigma^* \rightarrow (\Sigma^2)^*$, $g(x) = f(x) \Rightarrow g$ bijectivă! % pt ca e inversabila (exista inversul si este unic)
\end{obs}
Vom lucra cu funcția $g$.
}
\end{frame}


\begin{frame}
\frametitle{Descompunerea}

\only<1>{
\centering
\begin{forest}
for tree={
    grow=south,
    edge=->, minimum size=3ex, inner sep=1pt,
    s sep=7mm
}
[13112221
    [1113213211
        [31131211131221 
            [13211311123113112211
            ]
        ]
    ]
]
\end{forest}
}

\only<2>{
\centering
\begin{forest}
for tree={
    grow=south,
    edge=->, minimum size=3ex, inner sep=1pt,
    s sep=7mm
}
[13112221
    [\textcolor{blue}{11132}
        [\textcolor{blue}{311312}
            [\textcolor{blue}{1321131112}
            ]
        ]
    ]
    [\textcolor{red}{13211}
        [\textcolor{red}{11131221}
            [\textcolor{red}{3113112211}
            ]
        ]
    ]
]
\end{forest}
}

\only<3->{
\begin{itemize}
    \item<3-> $f^n(LR) = f^n(L)f^n(R)$
    \item<4-> Ultima cifră a lui $L_n \ne$ prima cifră a lui $R_n$, $\forall n \ge n_0$.
    \item<5-> Notație: $z = x.y$ %adica z se imparte in x si y
\end{itemize}
}
\end{frame}

\begin{frame}
\frametitle{Istoric}
\begin{columns}
\column{0.5\textwidth}
Rezultate:
\begin{itemize}
    \item<1->
\end{itemize}
\column{0.5\textwidth}
\centering
\begin{figure}
    \includegraphics[scale=0.12]{images/conway.jpg}
    \caption*{\rmfamily John Conway (1937-2020)}
\end{figure}
\end{columns}
\end{frame}

\section{Despre ETF-uri}
\begin{frame} 
\frametitle{Ce este un ETF?}
\begin{itemize}
    \item<2-> \textit{Exchange-traded fund}
    \item<3-> Fond de investiții listat și tranzacționat pe piețele bursiere
    \item<4-> Adesea urmăresc un indice bursier
    \item<5-> iShares Core S\&P 500 ETF (IVV)
\end{itemize}
\end{frame}

\begin{frame}
\frametitle{Avantaje ale ETF-urilor}
\begin{itemize}
    \item<2-> Diversitate $\rightarrow$ Risc scăzut
    \item<3-> Administrare pasivă, costuri administrative mici $\rightarrow$ Rată a cheltuielilor mică
    \item<4-> Lichiditate, eficiență
\end{itemize}
\end{frame}

\section{Cum funcționează aplicația?}
\begin{frame}
\frametitle{Preluarea datelor}
\begin{itemize}
    \item<2-> Yahoo Finance (API -- biblioteca \texttt{yfinance})
    \item<3-> Adaptare la diferite calendare bursiere prin interpolare (biblioteca \texttt{pandas})
    \item<4-> Trecere la euro 
\end{itemize}
\end{frame}

\begin{frame}
    \frametitle{Măsurarea performanței unui portofoliu}
\begin{itemize}
    \item<2->$p(i, t) = $ prețul de închidere al activului $i$ la timpul (ziua) $t$.
    \item<3->Rata rentabilității: $\displaystyle r(i, t) = \frac{p(i, t) - p(i, t-1)}{p(i, t-1)} = \frac{p(i, t)}{p(i, t-1)} - 1$.
    \item<4->Ponderi (weights): $w(i, t) \in [0, 1]$ și $\displaystyle \sum_i w(i,t) = 1$ (prezise de rețeaua neuronală).
    \item<5->Portofoliul realizat: $\displaystyle R(t) = \sum_i w(i,t)\cdot r(i,t)$.
\end{itemize}
\end{frame}

\begin{frame}
\frametitle{Raportul Sharpe}
 \begin{itemize}
    \item<2-> raportează performanța netă a portofoliului la riscul asumat \\($\approx$ abaterea standard)
    \only<3>{\item $\displaystyle L = \frac{\E[R(t)]}{\sigma} = \frac{\E[R(t)]}{\sqrt{\E[R^2(t)] - \E[R(t)]^2}}$}
    \only<4>{\item $\displaystyle L = \frac{\E[R(t)]}{\sigma} = \frac{\E[R(t)]}{\sqrt{\E[R^2(t)] - \E[R(t)]^2}} \cdot \sqrt{252}$}
\end{itemize}
\end{frame}

\begin{frame}
\frametitle{Softmax}
\begin{itemize}
    \item<1-> Modelul returnează ponderile care maximizează raportul Sharpe ($L$).
    \item<2-> Nu satisfac neapărat $w(i, t) \ge 0$ și $\displaystyle \sum_i w(i,t) = 1$!
    \item<3-> $\displaystyle \operatorname{softmax}(w(i, t)) = \frac{e^{w(i, t)}}{\sum\limits_{j=1}^n e^{w(j, t)}}$.
\end{itemize}
\end{frame}

\begin{frame}
\frametitle{Rețeaua neuronală}
\begin{itemize}
    \item<2-> 3 straturi:
    \begin{itemize}
        \item<3-> LSTM (32 neuroni, dropout: 0,2)
        \item<4-> Flatten
        \item<5-> Dense (Softmax)
    \end{itemize}
    \item<6-> Window-uri de câte 200 zile
    \item<7-> 80\% train, 20\% test
\end{itemize}
\end{frame}

\begin{frame}
\frametitle{Comunicare frontend -- backend}
\begin{itemize}
    \item<1-> API scris în biblioteca \texttt{fastapi}
    \item<2-> \texttt{index.html/process}
    \item<3-> \texttt{index.html/get\_etfs}
    \item<4-> \texttt{index.html/get\_etf\_history?etf=EPOL}
\end{itemize}
\end{frame}

\begin{frame}
\frametitle{Pagina web}
\begin{itemize}
    \item Grafice interactive în \texttt{plotly}
\end{itemize}
\end{frame}

\section{Limitări, posibilități de dezvoltare}
\begin{frame}
\frametitle{Limitări, posibilități de dezvoltare}
\begin{itemize}
    \item<1-> Anualizarea raportului Sharpe
    \item<2-> Mai multe ETF-uri listate la aceeași bursă
    \item<3-> Dezvoltare UI (ex. dropdown)
    \item<4-> Formulă alternativă pentru portofoliul realizat (volatilitate!) \\
    $\displaystyle R(t) = \sum_i \frac{\sigma_{tgt}}{\sigma(i,t-1)}w(i,t-1)\cdot r(i,t) 
    - C \cdot\sum_i\left|\frac{\sigma_{tgt}}{\sigma(i,t-1)}w(i,t-1) - \frac{\sigma_{tgt}}{\sigma(i,t-2)}w(i,t-2)\right|$
\end{itemize}
\end{frame}

\section{Vă mulțumim!}
\end{document}
