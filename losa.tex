\documentclass[notheorems]{beamer}
\usetheme{Madrid}
\usecolortheme{spruce}

\usepackage{graphicx} % Required for inserting images
\usepackage[romanian]{babel}
\usepackage[center]{caption}
\usepackage{float}
\usepackage{amsmath, amssymb}
\usepackage{enumitem}
\usepackage{mathtools}
\usepackage[T1]{fontenc}
\usepackage{ragged2e}
\usepackage[edges]{forest}
\usepackage{expl3}

\theoremstyle{definition}
\newtheorem*{obs}{Observație.}
\newtheorem{theorem}{Teoremă.}
\newtheorem{proposition}{Propoziție.}
\newtheorem{corollary}{Corolar.}
\newtheorem{definition}{Definiție.}
\newtheorem{lemma}{Lemă.}

\AtBeginSection[]{
  \begin{frame}
  \vfill
  \centering
  \begin{beamercolorbox}[sep=8pt,center,shadow=true,rounded=true]{title}
    \usebeamerfont{title}\insertsectionhead\par%
  \end{beamercolorbox}
  \vfill
  \end{frame}
}

\setlist[enumerate]{label=(\alph*), leftmargin=1.3cm}
\setlist[itemize,1]{label=\usebeamertemplate{itemize item}}
\setlist[itemize,2]{label=\usebeamertemplate{itemize subitem}}
\setlist[itemize,3]{label=\usebeamertemplate{itemize subsubitem}}

% \setbeamertemplate{items}[triangle]

\title[Șiruri audioactive]{\textbf{Șiruri audioactive}}
\author[Patrichi, Mureanu]{
    Răzvan-Anton Mureanu \texttt{\small<razvan.mureanu@cnmbct.ro>}
    \texorpdfstring{\\}{}
    Ștefan Patrichi \texttt{\small<stefan.patrichi.07@cnmbct.ro>}
}

\institute[CNMB CT]{
    \normalsize
    Colegiul Național „Mircea cel Bătrân” Constanța
}

\date[CITYINNOEDU]{AI + $\alpha$ + Z = Matematica noilor generații\\8 noiembrie 2025}

% Usual LaTeX font (math)
\usefonttheme[onlymath]{serif}

\begin{document}

% Usual LaTeX font (text)
\rmfamily

\justifying
\frame{\titlepage}

\Large
\begin{frame}
\frametitle{Care este următorul termen?}
\only<1>{
    \begin{center}
    1\\
    11\\
    21\\
    1211\\
    111221\\
    312211\\
    13112221\\
    1113213211\\
    \end{center}
}
\only<2>{
    \huge
    \begin{center}
        111221\\
        312211\\
    \end{center}
}
\only<3>{
    \huge
    \begin{center}
        \textcolor{blue}{111}\textcolor{red}{22}\textcolor{teal}{1}\\
        3\textcolor{blue}{1} 2\textcolor{red}{2} 1\textcolor{teal}{1}\\
    \end{center}
}
\end{frame}

\begin{frame}
\frametitle{Monoid liber}
    \begin{itemize}
        % nu lucram cu numere, ci cu cuvinte
        % notiunea de alfabet, cuvant, concatenare
        % notatie multiplicativa pt operatia de concatenare
        \item<1-> Alfabet: $\Sigma = \{1, 2, 3, 4, 5, 6, 7, 8, 9\}$
        \item<2-> Cuvânt: $\alpha = a_1a_2\ldots a_k$ cu $a_i \in \Sigma$
        \item<3-> Lungimea cuvântului: $|\alpha| = k$
        \item<4-> Concatenarea: $\alpha = a_1a_2\ldots a_k$, $\beta = b_1b_2\ldots b_l$\\
        Atunci $\alpha\cdot\beta = \alpha\beta = a_1a_2\ldots a_kb_1b_2\ldots b_l$
        \item<5-> Cuvânt vid: $\varepsilon$ cu $|\varepsilon| = 0$ și $\alpha\varepsilon = \varepsilon\alpha = \alpha \ \forall\alpha$
        \item<6-> Mulțimea cuvintelor: \\${\Sigma^* = \{a_1a_2\ldots a_k \mid a_i \in \Sigma, k \in \mathbb{N}\}}$
        \item<7-> $(\Sigma^*, \cdot)$ = \textbf{monoidul liber} generat de mulțimea $\Sigma$
    \end{itemize}
\end{frame}

\begin{frame}
\frametitle{Convenții pentru notația multiplicativă}
\begin{itemize}
    \item<1-> Notație: $\displaystyle \underbrace{aa\ldots a}_{\text{de $m$ ori}}\underbrace{bb\ldots b}_{\text{de $n$ ori}} = a^m b^n$
    \item<2-> $a^ma^n = a^{m+n}$
    \item<3-> Exemplu: $111221 = 1^3 2^2 1^1$ \only<3-> {$\rightarrow 312211$}
    \item<4-> Exemplu: $3333333333 = 3^{10}$ \only<4-> {$\rightarrow 103$} % nu e ok, definim acum unde e ok
\end{itemize}
\end{frame}

\begin{frame}{Terminologie}
\only<1>{
    \begin{definition}
        \rmfamily
        Un șir care evoluează după regula descrisă anterior se numește \textit{șir look-and-say} (\textit{privește-și-spune}).
    \end{definition}
    \begin{definition}
        \rmfamily
        Primul termen al unui șir look-and-say se numește \textit{origine}.
    \end{definition}
}
\only<2>{
    \begin{definition}
        \rmfamily
        Un șir look-and-say a cărui origine nu conține nicio înșiruire de mai mult de $9$ cifre identice adiacente se numește \textit{șir compact}.
    \end{definition}
    \begin{definition}
        \rmfamily
        Notăm cu $\mathcal{C} \subset \Sigma^*$ mulțimea tuturor cuvintelor care apar în șiruri compacte, excluzând originea.
    \end{definition}
}
% eu am ales un nume in directia audio... spre deosebire de conway (cel care a studiat sirul), care a optat pentru o analogie cu chimia, si vom vedea imediat de ce
\only<3>{
    \begin{definition}
        \rmfamily
        Un șir look-and-say a cărui origine:
        \begin{enumerate}
            \item nu conține nicio cifră mai mare decât $3$ și
            \item nu conține nicio înșiruire de mai mult de $3$ cifre identice adiacente
        \end{enumerate}
        se numește \textit{șir armonic}. % aici principalul exemplu de sir cu care vom lucra este cel cu s_0 = 1
    \end{definition}
    \begin{definition}
        \rmfamily
        Notăm cu $\mathcal{A} \subset \mathcal{C} \subset \Sigma^*$ mulțimea tuturor cuvintelor care apar în șiruri armonice, excluzând originea.
    \end{definition}
}
\end{frame}

\begin{frame}
    \frametitle{Funcția de tranziție}
    \begin{itemize}
        \item<1-> $LS : \mathcal{C} \rightarrow \mathcal{C}$, $\boxed{LS(a_1^{n_1}a_2^{n_2}\ldots a_k^{n_k}) = n_1 a_1 n_2 a_2 \ldots n_k a_k}$ % stim ca n-urile sunt toate cifre nenule
        \item<2-> Condiție: \textcolor{red}{$a_i \ne a_{i+1}$} $\left(a_i^m a_i^n = a_i^{m+n}\right)$
        \item<3-> $LS(777777) = 67$, nu $2747$ sau $5717$.
    \end{itemize}
\end{frame}

\begin{frame}
\frametitle{Proprietăți de bază}
\only<1-3>{
    \normalsize
    \onslide<1->{
        \begin{proposition}
            \rmfamily
        Mulțimea $\mathcal{C}$ coincide cu mulțimea cuvintelor $a_1a_2\ldots a_k$ pentru care:
        \begin{enumerate}
            \item $2 \mid k$,
            \item $a_{2i} \ne a_{2i+2}$.
        \end{enumerate}
        \end{proposition}
    }
    \onslide<2->{
        \begin{proof}
            Pentru incluziunea $\supset$, fie $a_1b_1a_2b_2\ldots a_kb_k$ un cuvânt astfel încât $b_i \ne b_{i+1}$. Atunci putem să-i construim inversul, $b_1^{a_1}b_2^{a_2}\ldots b_k^{a_k}$.
        \end{proof}
    }
    \onslide<3->{
        \begin{obs}
        \rmfamily
        Am demonstrat că funcția $f$ este bijectivă!
        \end{obs}
    }
}
\only<4>{
\begin{proposition}
\rmfamily
Mulțimea $\mathcal{A}$ coincide cu mulțimea cuvintelor $a_1a_2\ldots a_k$ pentru care:
\begin{enumerate}
    \item $2 \mid k$,
    \item $a_{2i} \ne a_{2i+2}$ și
    \item $a_i \in \{1,2,3\}$.
\end{enumerate}
\end{proposition}% Vom lucra cu funcția $g$.
}
\end{frame}


\begin{frame}
\frametitle{Proprietăți de bază -- Descompunerea}

\only<1>{
\centering
\begin{forest}
for tree={
    grow=south,
    edge=->, minimum size=3ex, inner sep=1pt,
    s sep=7mm
}
[13112221
    [1113213211
        [31131211131221 
            [13211311123113112211
            ]
        ]
    ]
]
\end{forest}
}

\only<2>{
\centering
\begin{forest}
for tree={
    grow=south,
    edge=->, minimum size=3ex, inner sep=1pt,
    s sep=7mm
}
[13112221
    [\textcolor{blue}{11132}
        [\textcolor{blue}{311312}
            [\textcolor{blue}{1321131112}
            ]
        ]
    ]
    [\textcolor{red}{13211}
        [\textcolor{red}{11131221}
            [\textcolor{red}{3113112211}
            ]
        ]
    ]
]
\end{forest}
}

\only<3->{
\begin{itemize}
    \item<3-> $LS^n(S \cdot D) = LS^n(S)\cdot LS^n(D)$ % iterata
    \item<4-> Ultima cifră a lui $LS^n(S) \ne$ prima cifră a lui $LS^n(D)$, pentru $\forall n \ge n_0$.
    % \item<5-> Notație: $z = x.y$ %adica z se imparte in x si y
    \item<5-> Dacă $y$ se descompune în $x_1$, $x_2$, \ldots, $x_k$, atunci:
    \begin{align*}
        \onslide<5->{LS^n(y) &= \prod_{k=1}^n LS^n(x_k)}
        \onslide<6->{\\ |LS^n(y)| &= \sum_{k=1}^n |LS^n(x_k)|}
    \end{align*}
\end{itemize}
}
\end{frame}

\begin{frame}
\frametitle{Istoric}
\begin{columns}
\column{0.6\textwidth}
Rezultate:
\begin{itemize}
    \item<only@1> Modul în care se repetă începutul și sfârșitul numerelor dintr-un șir look-and-say
    \item<only@2> Condiții necesare și suficiente pentru descompunere
    \item<only@3> Teoremă: \textbf{Există 92 de \textit{elemente atomice} astfel încât orice origine se descompune eventual într-o succesiune de elemente atomice} % garantat de la al 8 lea incolo % restul sunt compusi
    \item<only@4> \textbf{Rata asimptotică de creștere a lungimii numerelor dintr-un șir look-and-say}, \\\vspace{2mm}${\displaystyle \lambda = \lim_{n\to\infty}\frac{|s_{n+1}|}{|s_n|}}$
\end{itemize}
\column{0.4\textwidth}
\centering
\begin{figure}
    \includegraphics[scale=0.105]{images/conway.jpg}
    \caption*{\rmfamily John Conway (1937-2020)}
\end{figure}
\end{columns}
\end{frame}

\begin{frame}{Metodă de calcul pentru $\lambda$}
    \only<1-3>{
        \begin{columns}
            \column{0.5\textwidth}
                \begin{itemize}
                    \item<1-> $e_i$ = cele $d$ elemente atomice
                    \item<2-> $t_{ij}$ = de câte ori apare $e_i$ în regula de evoluție a lui $e_j$ % adica descompunerea lui $LS(e_j)$
                    \item<3-> $f_i$ = de câte ori apare $e_i$ în descompunerea originii șirului
                \end{itemize}
            \column{0.5\textwidth}
                \centering
                \begin{figure}
                    \includegraphics[scale=0.24]{images/periodic_table.png}
                    \caption*{\rmfamily Fragment din \textit{Tabelul periodic} al celor $d = 92$ elemente}
                \end{figure}
        \end{columns}
    }
    \only<4-10>{
        \begin{itemize}
            \item<4-> Fie $n_0$ astfel încât $LS^{n_0}(s_0) = s_{n_0}$ se descompune în elemente atomice.
            \onslide<5->{\begin{align*}
             |s_{n_0+1}| &= \sum_{i = 1}^{d} (\text{nr. apariții $e_i$ în $LS(s_{n_0})$}) \cdot |e_i|
             \only<6>{\\ &=  \sum_{i = 1}^{d} \left(\sum_{j=1}^d t_{ij}f_j\right) \cdot |e_i| }
             \only<7-8>{\\ &=  \sum_{i = 1}^{d} \left(\textcolor{red}{\sum_{j=1}^d t_{ij}f_j}\right) \cdot |e_i| }
             \only<9->{\\ &= (\mathbf{Tf}) \cdot \mathbf{l} = \mathbf{l}^\top(\mathbf{Tf})}
            \end{align*}
            \vspace{-4mm}
             \only<8-9>{
             \begin{columns}
                \column{0.25\textwidth}
                    \begin{align*}
                        \mathbf{T} = 
                        \begin{pmatrix}
                            t_{11} & t_{12} & \cdots & t_{1d} \\
                            t_{21} & t_{22} & \cdots & t_{2d} \\
                            \vdots & \vdots & \ddots & \vdots \\
                            t_{d1} & t_{d2} & \cdots & t_{dd} \\
                        \end{pmatrix}
                    \end{align*}
                \column{0.25\textwidth}
                    \begin{align*}
                        \mathbf{f} = \begin{pmatrix}
                            f_1 \\ f_2 \\ \vdots \\ f_d
                        \end{pmatrix}
                    \end{align*}
                \column{0.25\textwidth}
                    \begin{align*}
                        \mathbf{l} = \begin{pmatrix}
                            |e_1| \\ |e_2| \\ \vdots \\ |e_d|
                        \end{pmatrix}
                    \end{align*}
             \end{columns}
             }
            }
            \item<10->{Analog, \begin{align*}
                |s_{n_0+k}| = (\mathbf{T}^k \mathbf{f}) \cdot \mathbf{l} = \mathbf{l}^\top(\mathbf{T}^k \mathbf{f})
            \end{align*}} % T - matricea de tranzitie
        \end{itemize}
    }
    \only<11-15>{
        \begin{itemize}
            \item<11-> $\lambda_i$, $\mathbf{v}_i$ = valori/vectori proprii ai matricei $\mathbf{T}$ % toti, nu doar cei asociati lui lambda
            \item<12-> $\lambda_p$, $\mathbf{v}_p$ = valoare/vector propriu Perron a matricei $\mathbf{T}$
            \item<13-> Pentru simplitate, $n_0 = 0$.
            \item<14-> Când $\mathbf{T}$ e diagonalizabilă,
            \begin{align*}
                \frac{1}{\lambda^n}\mathbf{T}^n\mathbf{f} &= \frac{1}{\lambda^n}\mathbf{T}^n\sum_{i=1}^{d}\alpha_i\mathbf{v}_i
                \onslide<14->{\\ &= \sum_{i=1}^{d}\alpha_i\left(\frac{\lambda_i}{\lambda}\right)^n\mathbf{v}_i}
                \onslide<15->{ \rightarrow \alpha_p \mathbf{v}_p}
            \end{align*} 
        \end{itemize}
    }
    \only<16->{
        \begin{itemize}
            \item<16-> Acum, 
            \begin{align*}
                \only<16>{\lim_{n\to\infty}\frac{|s_{n+1}|}{|s_n|} = \lim_{n\to\infty}\lambda_p\frac{\displaystyle\frac{1}{\lambda_p^{n+1}}\mathbf{T}^{n+1}\mathbf{f} \cdot \mathbf{l}}
                {\displaystyle\frac{1}{\lambda_p^{n}}\mathbf{T}^{n}\mathbf{f} \cdot \mathbf{l}}}
                \onslide<17->{\lim_{n\to\infty}\frac{|s_{n+1}|}{|s_n|} &= \lim_{n\to\infty}\lambda_p\frac{\textcolor{red}{\displaystyle\frac{1}{\lambda_p^{n+1}}\mathbf{T}^{n+1}\mathbf{f}} \cdot \mathbf{l}}
                {\displaystyle \textcolor{red}{\frac{1}{\lambda_p^{n}}\mathbf{T}^{n}\mathbf{f}} \cdot \mathbf{l}}}
                \onslide<18->{\\ &= \lambda_p\frac{\alpha_p \mathbf{v}_p \cdot \mathbf{l}}{\alpha_p \mathbf{v}_p \cdot \mathbf{l}} \\ &= \lambda_p \approx 1.303577269034\ldots}
            \end{align*}
        \end{itemize}
    }
\end{frame}

\section{Varianțiuni de șiruri look-and-say} % lucram cu primul termen mereu 1
\begin{frame}{Varianțiuni de șiruri look-and-say}
    \begin{itemize}
        \item $LS$ cu nr. de apariții postpus ($LS_{\texttt{post}}$)
        \begin{align*}
            LS_{\texttt{post}}(a_1^{n_1}a_2^{n_2}\ldots a_k^{n_k}) = a_1 n_1 a_2 n_2 \ldots a_k n_k
        \end{align*}
        \item $LS$ în baza 2 ($LS_{\texttt{b2}}$) % avand in vedere ca in \mathcal{A} avem doar cifre de 1,2,3 are rost sa ne uitam doar la bazele 2 si 3
        \item $LS$ cu cifre romane ($LS_{\texttt{roman}}$)
        \item $LS$ cu suprimarea cifrelor \textit{singure} ($LS_{\texttt{sng}}$)
        \begin{align*}
            LS_{\texttt{sng}}(a_1^{n_1}a_2^{n_2}\ldots a_k^{n_k}) = m_1 a_1 m_2 a_2 \ldots m_k a_k,
        \end{align*}
        unde:
        \begin{align*}
            m_i = \begin{cases}
            \varepsilon, & n_i = 1 \\
            n_i, &n_i \ge 2
            \end{cases}
        \end{align*}
    \end{itemize}
\end{frame}

\begin{frame}{Probleme de investigat}
    \begin{itemize}
        \item Elemente atomice
        \item Puncte fixe (de ex. $\varepsilon$ și $22$ pentru $LS$-ul obișnuit)
        \item Rata asimptotică de creștere a lungimii cuvintelor ($\lambda$)
        \item Stabilizarea (șirul $(s_n)$ constant de la un rang încolo)
        \item Alte proprietăți specifice (regularități/ciclicități)
    \end{itemize}
\end{frame}

\begin{frame}{$LS_{\texttt{post}}$}
    \only<1-2>{
    \begin{align*}
    LS&: 1, 11, 21, 1211, 111221, 312211, 13112221, \ldots \\
    LS_{\texttt{post}}&: 1, 11, 12, 1121, 122111, 112213, 12221131\ldots
    \end{align*}
    \onslide<2>{
        \begin{theorem}
            \rmfamily
            $LS_{\texttt{post}}^n(s_0) = R(LS^n(s_0))$, unde $R : \mathcal{C} \to \mathcal{C}$ este operatorul de \textit{răsturnare} a cuvintelor.
        \end{theorem}
    }
    }
    \only<3>{
        \centering
        \includegraphics[scale=0.33]{images/proof11.png}
    }
    \only<4>{
        \centering
        \includegraphics[scale=0.35]{images/proof12.png} % gemini similar, putin mai intuitiv, mai lung
    }
\end{frame}

\begin{frame}{$LS_{\texttt{b2}}$}
\only<1>{$LS_{\texttt{b2}}: 1, 11, 101, 111011, 11110101, 100110111011$}
% sunt 13 elemente atomice, chatgpt gaseste site ul lui johnston cu 10 pe net
% gemini halucineaza, da o lucrare inexistenta (cu 15 elemente atomice), intreb care lucrare, se redreseaza, mentioneaza lucrarea lui silke (11) si site ul lui njohnston gasit si de chatgpt
\only<2>{
    \centering
    \includegraphics[scale=0.4]{images/proof21.png} % corectez % inegalitate in loc de egalitate la log, faptul ca incepe mereu cu 1 deci blocurile vor fi 1 0 1 0 etc., faptul ca nu trb sa continue cu bloc de 0
}
\end{frame}

\begin{frame}{$LS_{\texttt{roman}}$}
\only<1>{% in primul rand numerele nu sunt numere valide cu cifre romane
    \begin{enumerate}[label=\arabic*.]
        \item i
        \item ii
        \item iii
        \item iiii
        \item ivi
        \item iiivii
        \item iiiiiviii
        \item viiviiii
        \item iviiiivivi
        \item iiiviviiviiivii
        \item iiiiiviiiviiiiviiiiiviii
        \item viiviiiiiviviivviiviiii
    \end{enumerate}
}
\only<2>{
      \begin{enumerate}[label=\arabic*.]
        \item i
        \item ii
        \item iii
        \item iiii
        \item \textcolor{red}{i}vi
        \item \textcolor{red}{iii}vii
        \item \textcolor{red}{iiiii}viii
        \item \textcolor{red}{}viiviiii
        \item \textcolor{red}{i}viiiivivi
        \item \textcolor{red}{iii}viviiviiivii
        \item \textcolor{red}{iiiii}viiiviiiiviiiiiviii
        \item \textcolor{red}{}viiviiiiiviviivviiviiii
    \end{enumerate}
}
\only<3>{
    \begin{tabular}{rl}
    1. &i\\
    5. &ivi\\
    9. &iviiiiviii\\
    13. &iviiiivviiviiiviiiiviiiiiviii\\
    17. &iviiiivviiiviiiiviiiiivviivviiviiiviiiivviiviiiviiiiiviiiivivi\\
    21. &iviiiivviiiviiiiiviviivviiiviiiiiviiiiiviviiviiiiivviiviviivv\ldots\\
    25. &iviiiivviiiviiiiivviiviiiviiiiiviiiiivviivviiiviiiiviiiviiiii\ldots\\
    29. &iviiiivviiiviiiiivviiiviiiiviiiiivviivviiiviiiiiviiiiivviiviv\ldots\\
    33. &iviiiivviiiviiiiivviiiviiiiiviviivviiiviiiiiviiiiivviivviiivi\ldots\\
    \end{tabular}
}
\end{frame}
\section{Vă mulțumim!}
\end{document}
