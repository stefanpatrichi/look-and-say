\documentclass[notheorems]{beamer}
\usetheme{Madrid}
\usecolortheme{spruce}

\usepackage{graphicx} % Required for inserting images
\usepackage[romanian]{babel}
\usepackage[center]{caption}
\usepackage{float}
\usepackage{amsmath, amssymb}
\usepackage{enumitem}
\usepackage{mathtools}
\usepackage[T1]{fontenc}
\usepackage{ragged2e}
\usepackage[edges]{forest}

\theoremstyle{definition}
\newtheorem*{obs}{Observație.}
\newtheorem{theorem}{Teoremă.}
\newtheorem{proposition}{Propoziție.}
\newtheorem{corollary}{Corolar.}
\newtheorem{definition}{Definiție.}
\newtheorem{lemma}{Lemă.}

\AtBeginSection[]{
  \begin{frame}
  \vfill
  \centering
  \begin{beamercolorbox}[sep=8pt,center,shadow=true,rounded=true]{title}
    \usebeamerfont{title}\insertsectionhead\par%
  \end{beamercolorbox}
  \vfill
  \end{frame}
}

\setlist[enumerate]{label=(\alph*), leftmargin=1.3cm}
\setlist[itemize,1]{label=\usebeamertemplate{itemize item}}
\setlist[itemize,2]{label=\usebeamertemplate{itemize subitem}}
\setlist[itemize,3]{label=\usebeamertemplate{itemize subsubitem}}

% \setbeamertemplate{items}[triangle]

\title[Șiruri audioactive]{\textbf{Șiruri audioactive}}
\author[Patrichi, Mureanu]{
    Răzvan-Anton Mureanu \texttt{\small<razvan.mureanu@cnmbct.ro>}
    \texorpdfstring{\\}{}
    Ștefan Patrichi \texttt{\small<stefan.patrichi.07@cnmbct.ro>}
}

\institute[CNMB CT]{
    \normalsize
    Colegiul Național „Mircea cel Bătrân” Constanța
}

\date[CITYINNOEDU]{AI + $\alpha$ + Z = Matematica noilor generații\\8 noiembrie 2025}

% Usual LaTeX font (math)
\usefonttheme[onlymath]{serif}

\begin{document}

% Usual LaTeX font (text)
\rmfamily

\justifying
\frame{\titlepage}

\Large
\begin{frame}
\frametitle{Care este următorul termen?}
\only<1>{
    \begin{center}
    1\\
    11\\
    21\\
    1211\\
    111221\\
    312211\\
    13112221\\
    1113213211\\
    \end{center}
}
\only<2>{
    \huge
    \begin{center}
        111221\\
        312211\\
    \end{center}
}
\only<3>{
    \huge
    \begin{center}
        \textcolor{blue}{111}\textcolor{red}{22}\textcolor{teal}{1}\\
        3\textcolor{blue}{1} 2\textcolor{red}{2} 1\textcolor{teal}{1}\\
    \end{center}
}
\end{frame}

\begin{frame}
\frametitle{Monoid liber}
    \begin{itemize}
        % nu lucram cu numere, ci cu cuvinte
        % notiunea de alfabet, cuvant, concatenare
        % notatie multiplicativa pt operatia de concatenare
        \item<1-> Alfabet: $\Sigma = \{1, 2, 3, 4, 5, 6, 7, 8, 9\}$
        \item<2-> Cuvânt: $\alpha = a_1a_2\ldots a_k$ cu $a_i \in \Sigma$
        \item<3-> Mulțimea cuvintelor: \\${\Sigma^* = \{a_1a_2\ldots a_k \mid a_i \in \Sigma, k \in \mathbb{N}\}}$
        \item<4-> Lungimea cuvântului: $|\alpha| = k$
        \item<5-> Concatenarea: $\alpha = a_1a_2\ldots a_k$, $\beta = b_1b_2\ldots b_l$\\
        Atunci $\alpha\cdot\beta = \alpha\beta = a_1a_2\ldots a_kb_1b_2\ldots b_l$
        \item<6-> $(\Sigma^*, \cdot)$ = \textbf{monoidul liber} generat de mulțimea $\Sigma$
    \end{itemize}
\end{frame}

\begin{frame}
\frametitle{Convenții pentru notația multiplicativă}
\begin{itemize}
    \item<1-> Notație: $\displaystyle \underbrace{aa\ldots a}_{\text{de $m$ ori}}\underbrace{bb\ldots b}_{\text{de $n$ ori}} = a^m b^n$
    \item<2-> $a^ma^n = a^{m+n}$
    \item<3-> Exemplu: $111221 = 1^3 2^2 1^1$ \only<3-> {$\rightarrow 312211$}
    \item<4-> Exemplu: $3333333333 = 3^{10}$ \only<5-> {$\rightarrow 103$} % nu e ok, definim acum unde e ok
\end{itemize}
\end{frame}

\begin{frame}{Terminologie}
\only<1>{
    \begin{definition}
        \rmfamily
        Un șir care evoluează după regula descrisă anterior se numește \textit{șir look-and-say} (\textit{privește-și-spune}).
    \end{definition}
    \begin{definition}
        \rmfamily
        Primul termen al unui șir look-and-say se numește \textit{origine}.
    \end{definition}
}
\only<2>{
    \begin{definition}
        \rmfamily
        Un șir look-and-say a cărui origine nu conține nicio înșiruire de mai mult de $9$ cifre identice adiacente se numește \textit{șir compact}.
    \end{definition}
    \begin{definition}
        \rmfamily
        Notăm cu $\mathcal{C} \subset \Sigma^*$ mulțimea tuturor cuvintelor care apar în șiruri compacte, excluzând originea.
    \end{definition}
}
% eu am ales un nume in directia audio... spre deosebire de conway (cel care a studiat sirul), care a optat pentru o analogie cu chimia, si vom vedea imediat de ce
\only<3>{
    \begin{definition}
        \rmfamily
        Un șir look-and-say a cărui origine:
        \begin{enumerate}
            \item nu conține nicio cifră mai mare decât $3$ și
            \item nu conține nicio înșiruire de mai mult de $3$ cifre identice adiacente
        \end{enumerate}
        se numește \textit{șir armonic}. % aici principalul exemplu de sir cu care vom lucra este cel cu s_0 = 1
    \end{definition}
    \begin{definition}
        \rmfamily
        Notăm cu $\mathcal{A} \subset \mathcal{C} \subset \Sigma^*$ mulțimea tuturor cuvintelor care apar în șiruri armonice, excluzând originea.
    \end{definition}
}
\end{frame}

\begin{frame}
    \frametitle{Funcția de tranziție}
    \begin{itemize}
        \item<1-> $f : \mathcal{C} \rightarrow \mathcal{C}$, $\boxed{f(a_1^{n_1}a_2^{n_2}\ldots a_k^{n_k}) = n_1 a_1 n_2 a_2 \ldots n_k a_k}$ % stim ca n-urile sunt toate cifre nenule
        \item<2-> Condiție: \textcolor{red}{$a_i \ne a_{i+1}$} $\left(a_i^m a_i^n = a_i^{m+n}\right)$
        \item<3-> $f(777777) = 67$, nu $2747$ sau $5717$.
    \end{itemize}
\end{frame}

\begin{frame}
\frametitle{Proprietăți de bază}
\only<1-3>{
    \normalsize
    \onslide<1->{
        \begin{proposition}
            \rmfamily
        Mulțimea $\mathcal{C}$ coincide cu mulțimea cuvintelor $a_1a_2\ldots a_k$ pentru care:
        \begin{enumerate}
            \item $2 \mid k$,
            \item $a_{2i} \ne a_{2i+2}$.
        \end{enumerate}
        \end{proposition}
    }
    \onslide<2->{
        \begin{proof}
            Pentru incluziunea $\supset$, fie $a_1b_1a_2b_2\ldots a_kb_k$ un cuvânt astfel încât $b_i \ne b_{i+1}$. Atunci putem să-i construim inversul, $b_1^{a_1}b_2^{a_2}\ldots b_k^{a_k}$.
        \end{proof}
    }
    \onslide<3->{
        \begin{obs}
        \rmfamily
        Am demonstrat că funcția $f$ este bijectivă!
        \end{obs}
    }
}
\only<4>{
\begin{proposition}
\rmfamily
Mulțimea $\mathcal{A}$ coincide cu mulțimea cuvintelor $a_1a_2\ldots a_k$ pentru care:
\begin{enumerate}
    \item $2 \mid k$,
    \item $a_{2i} \ne a_{2i+2}$ și
    \item $a_i \in \{1,2,3\}$.
\end{enumerate}
\end{proposition}% Vom lucra cu funcția $g$.
}
\end{frame}


\begin{frame}
\frametitle{Proprietăți de bază -- Descompunerea}

\only<1>{
\centering
\begin{forest}
for tree={
    grow=south,
    edge=->, minimum size=3ex, inner sep=1pt,
    s sep=7mm
}
[13112221
    [1113213211
        [31131211131221 
            [13211311123113112211
            ]
        ]
    ]
]
\end{forest}
}

\only<2>{
\centering
\begin{forest}
for tree={
    grow=south,
    edge=->, minimum size=3ex, inner sep=1pt,
    s sep=7mm
}
[13112221
    [\textcolor{blue}{11132}
        [\textcolor{blue}{311312}
            [\textcolor{blue}{1321131112}
            ]
        ]
    ]
    [\textcolor{red}{13211}
        [\textcolor{red}{11131221}
            [\textcolor{red}{3113112211}
            ]
        ]
    ]
]
\end{forest}
}

\only<3->{
\begin{itemize}
    \item<3-> $f^n(LR) = f^n(L)f^n(R)$ % iterata
    \item<4-> Ultima cifră a lui $f^n(L) \ne$ prima cifră a lui $f^n(R)$, pentru $\forall n \ge n_0$.
    % \item<5-> Notație: $z = x.y$ %adica z se imparte in x si y
\end{itemize}
}
\end{frame}

\begin{frame}
\frametitle{Istoric}
\begin{columns}
\column{0.6\textwidth}
Rezultate:
\begin{itemize}
    \item<only@1> Modul în care se repetă începutul și sfârșitul numerelor dintr-un șir look-and-say
    \item<only@2> Condiții necesare și suficiente pentru descompunere
    \item<only@3> Teoremă: \textbf{Există 92 de \textit{elemente atomice} astfel încât orice origine se descompune eventual într-o succesiune de elemente atomice}
    \item<only@4> Rata asimptotică de creștere a lungimii numerelor dintr-un șir look-and-say
\end{itemize}
\column{0.4\textwidth}
\centering
\begin{figure}
    \includegraphics[scale=0.105]{images/conway.jpg}
    \caption*{\rmfamily John Conway (1937-2020)}
\end{figure}
\end{columns}
\end{frame}

\section{Vă mulțumim!}
\end{document}
